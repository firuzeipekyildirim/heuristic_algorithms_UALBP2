\documentclass[12pt,a4paper]{article}

% =========================
% Temel paket yüklemeleri
% =========================
\usepackage[utf8]{inputenc}
\usepackage[T1]{fontenc}
\usepackage[english]{babel}

% Sayfa yapısı ve kenar boşlukları
\usepackage[margin=2.5cm]{geometry}

% Satır aralığını 1.5 yapıyorum
\usepackage{setspace}
\onehalfspacing

% Matematik ortamları için gerekli paketler
\usepackage{amsmath, amssymb}

% Tablo paketleri
\usepackage{booktabs}
\usepackage{longtable}
\usepackage{array}
\usepackage{multicol}

% Görselleri kullanmak için
\usepackage{graphicx}
\usepackage{float}
\usepackage{caption}

% Linkleri düzenlemek için
\usepackage{hyperref}

% Algoritma paketleri
\usepackage{algorithm}
\usepackage{algpseudocode}

% =========================
% Renk tanımları
% =========================
\usepackage{xcolor}
\definecolor{richmaroonxd}{RGB}{186, 52, 90}

% =========================
% Algoritma anahtar kelimelerinin renk ve yazım stilini ayarlıyorum
% =========================
\algrenewcommand\algorithmicprocedure{\textbf{\textcolor{richmaroonxd}{Procedure}}}
\algrenewcommand\algorithmicwhile{\textbf{\textcolor{richmaroonxd}{While}}}
\algrenewcommand\algorithmicfor{\textbf{\textcolor{richmaroonxd}{For}}}
\algrenewcommand\algorithmicforall{\textbf{\textcolor{richmaroonxd}{For all}}}
\algrenewcommand\algorithmicif{\textbf{\textcolor{richmaroonxd}{If}}}
\algrenewcommand\algorithmicelse{\textbf{\textcolor{richmaroonxd}{Else}}}
\algrenewcommand\algorithmicreturn{\textbf{\textcolor{richmaroonxd}{Return}}}
\algrenewcommand\algorithmicend{\textbf{\textcolor{richmaroonxd}{End}}}

% Algoritma input-output başlıkları
\algnewcommand\algorithmicinput{\textbf{\textcolor{richmaroonxd}{Input:}}}
\algnewcommand\Input{\item[\algorithmicinput]}
\algnewcommand\algorithmicoutput{\textbf{\textcolor{richmaroonxd}{Output:}}}
\algnewcommand\Output{\item[\algorithmicoutput]}

% Fonksiyon isimleri için ufak bir kolaylık
\newcommand{\Func}[1]{\textcolor{richmaroonxd}{#1}}
\usepackage{listings}
\usepackage{xcolor}

% Atom One Light renkleri
\definecolor{atom-bg}{HTML}{FAFAFA}
\definecolor{atom-blue}{HTML}{007ACC}
\definecolor{atom-purple}{HTML}{AF00DB}
\definecolor{atom-green}{HTML}{098658}
\definecolor{atom-gray}{HTML}{6A6A6A}
\definecolor{atom-orange}{HTML}{D16969}

\lstdefinestyle{atomone}{
    backgroundcolor=\color{atom-bg},
    basicstyle=\ttfamily\small,
    keywordstyle=\color{atom-blue}\bfseries,
    stringstyle=\color{atom-green},
    commentstyle=\color{atom-gray}\itshape,
    numberstyle=\tiny\color{atom-gray},
    numbers=left,
    stepnumber=1,
    numbersep=10pt,
    frame=single,
    rulecolor=\color{atom-gray},
    tabsize=4,
    showstringspaces=false,
    breaklines=true,
    emph={
        True, False, None,
        int, float, str, list, dict, set, tuple,
        range, len, print, return
    },
    emphstyle=\color{atom-purple}\bfseries,
    moredelim=**[is][\color{atom-orange}]{@}{@}  % özel vurgular için
}

% Kolay kullanım komutu
\newcommand{\pythoncode}[2]{
\begin{lstlisting}[style=atomone,caption={#1}]
#2
\end{lstlisting}
}

% =========================
% Bölüm numaralandırma stilini ayarlıyorum
% =========================
\renewcommand{\thesection}{\arabic{section}}
\renewcommand{\thesubsection}{\arabic{section}.\arabic{subsection}}

% =========================
% Başlık bilgileri
% =========================
\title{\textbf{Solving the U-shaped Assembly Line Balancing Problem Type-II\\Using a Genetic Algorithm}}

\author{
Firuze İpek Yıldırım \quad 2220469028 \\
Mustafa Alp Ulaş \quad 2210469034 \\
Beril Yıldız \quad 2230469107 \\
Pınar Ece Pank \quad 2240469085 \\
Tarık Buğra Birinci \quad 2210469046 \\[6pt]
\textit{Department of Industrial Engineering, Hacettepe University}
}

\date{Project Final Report\\December 2025}

% =========================
% DOKÜMAN BAŞLANGICI
% =========================
\begin{document}


% KAPAK SAYFASI

\begin{titlepage}
    \centering
    \vspace*{2cm}

    {\LARGE \textbf{Solving the U-Shaped Assembly Line Balancing Problem Type-II \\[2mm]
    Using a Genetic Algorithm}} \par

    \vspace{1.4cm}

    {\large \textbf{Project Final Report}} \\[0.4cm]
    {\large EMU427 -- Heuristic Methods for Optimization} \\[0.3cm]
    {\large Department of Industrial Engineering \\ Hacettepe University}

    \vspace{2cm}

    {\large \textbf{Students}} \\[0.5cm]

    \begin{center}
        \begin{tabular}{l r}
            Firuze İpek Yıldırım & 2220469028 \\
            Mustafa Alp Ulaş     & 2210469034 \\
            Beril Yıldız         & 2230469107 \\
            Pınar Ece Pank       & 2240469085 \\
            Tarık Buğra Birinci  & 2210469046 \\
        \end{tabular}
    \end{center}

    \vspace{1.6cm}

    {\large \textbf{Instructor:} Prof. Çağrı Koç}

    \vfill
    {\large December 2025 \\ Ankara, Türkiye}
\end{titlepage}


% İÇİNDEKİLER

\tableofcontents
\newpage


% INTRODUCTION

\section{Introduction}
% Bu bölümde: U-shaped assembly line kavramı, literatür özeti,
%%%%%%%%%%% !!!!!!!!!!!!!!documentten copy paste yap ipek
% UALBP-II’nin neden önemli olduğu ve GA seçme motivasyonumuz anlatılacak.


% PROBLEM FORMULATION

\section{Problem Formulation}

\subsection{UALBP-II Definition}
% Burada problemi resmi olarak tanımlayacağım: amaç fonksiyonu, cycle time,
% sabit istasyon sayısı ve U-shaped yapıdaki yön değişimleri.

\subsection{Mathematical Model}
% Modelin değişkenlerini, parametrelerini ve kısıtlarını buraya yazıyorum.


% GENETIC ALGORITHM

\section{Description of the Genetic Algorithm}

\subsection{Representation}
% Burada GA içinde kullandığımız encoding yapısını açıklıyorum.

\subsection{Fitness Function}
% Fitness = cycle time. Ayrıca feasibility kontrolleri ve penalty mekanizması varsa belirtilecek.

\subsection{Selection, Crossover, Mutation}
% Operatörlerin detaylarını buraya yazıyorum.

\subsection{Repair and U-Shaped Decoding}
% U-shaped yapıyı düzeltmek için kullandığımız repair/decode adımlarını açıklıyorum.


% RESULTS

\section{Results}

\subsection{Parameter Tuning}
% Tuning denemelerimizi tablo ve grafiklerle raporlayacağım.

\subsection{Run-by-Run Performance}
% 10 run sonuçları, best/avg/worst değerlendirmesi.

\subsection{Best Solution Details}
% En iyi çözümün cycle time’ı, istasyon yükleri ve literatür karşılaştırması.


% SDG BÖLÜMÜ

\section{Contribution to Sustainable Development Goals}
% SDG 9, 12 ve 13 ile ilişkili kısmı buraya yazacağım.


% SONUÇ

\section{Conclusions}
% Çalışmanın ana bulguları ve gelecekte yapılabilecek geliştirmeler.


% KAYNAKLAR

\bibliographystyle{IEEEtran}
\bibliography{refs}


% EKLER

\appendix
\section{Python Source Code}
% Buraya GA ve tuning kodunu ekleyeceğim.
\begin{lstlisting}[style=atomone, caption={Simple Python Example}]
def add(a, b):
    return a + b

print(add(3, 5))
\end{lstlisting}

\end{document}
